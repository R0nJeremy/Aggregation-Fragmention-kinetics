\documentclass[aps,prl,preprint,groupedaddress,10pt]{revtex4-2}
\usepackage{notations}
\usepackage{tikz}


\begin{document}
\section{The mechanics of binary interactions}
Our system comprises an infinite number of mechanically identical, spherical particles 
known as monomers. These particles have masses $m_1$ and radii $R_1$. When two 
monomers collide, they lose a certain amount of impact energy and rebound with a 
coefficient of restitution $\eps$. If the impact energy is below a specific threshold 
value $\ltx{E}{imp}\leqslant\ltx{E}{agg}$, the monomers stick together due to surface 
forces like van der Waals forces, forming a larger aggregate particle with mass $m_2$ 
and radius $R_2$. This process is known as \emph{aggregation} and allows the creation 
of larger particles from individual monomers. A particle of mass $m_k$ is an aggregate
of $k$ monomers, hence $m_k=k\cdot m_1$. We assume that the aggregates remain 
spherically shaped, and their radii scale as $R_k\sim m_k^{1/3}\sim k^{1/3}$.

Conversely, there is another mechanism called \emph{fragmentation} that decreases the 
sizes of aggregates. If the impact energy is higher than a certain threshold value 
$\ltx{E}{imp}\geqslant\ltx{E}{frag}$, the colliding aggregates break into smaller pieces.
The size distribution of the fragmented pieces are difficult to model analytically, and
in this work we assume a simplistic model of fragmentation, called \emph{shattering}.
When two aggregates of masses $m_i$ and $m_j$ collide with a sufficient energy, both 
of them shatter into singular monomers, $m_i\to i\cdot m_1$ and $m_j\to j\cdot m_1$.

Let us consider a collision of two particles of masses $m_i$, $m_j$, and velocities 
$\bv_i$, $\bv_j$ and radii $R_i$, $R_j$ in the center of mass (CoM) frame, which moves 
with velocity $\bV=\mu_i\bv_i+\mu_j\bv_j$, where $\mu_i=m_i/M$, $\mu_j=m_j/M$ and 
$M=m_i+m_j$. The velocities of particles in the CoM are 
\begin{equation}
    \bv_{ci}=\mu_j\bg,\qquad\bv_{cj}=-\mu_i\bg,
\end{equation}
where $\bg=\bv_i-\bv_j$ is the relative velocity. Obviously, both particles in the CoM
frame have equal amounts of momenta, but oppositely directed
\begin{equation}
    \bp_{ci}=\mu\bg=\bp,\qquad\bp_{cj}=-\mu\bg=-\bp,
\end{equation}
where $\mu=m_j\mu_i=m_i\mu_j$ is the reduced mass. In the CoM frame, the relative velocity 
vector is 

The impact energy is then
\begin{equation}
    \ltx{E}{imp}=\frac{p^2}{2\mu}=\frac{\mu g^2}{2}.
\end{equation}

The outcome of the collision depends on the value of $\ltx{E}{imp}$. Let us discuss each 
case in more details.

\subsection{Restitution}
In general, when two particles collide, they rebound from each other and lose some amount
of energy, which is given by the restitution coefficient $\eps$. The post-collisional 
velocities of the colliding particles are given by
\begin{equation}
    \begin{split}
        \bv_i'&=\bv_i-\mu_j\pqty{1+\eps}\pqty{\bg\vdot\be}\be,\\
        \bv_j'&=\bv_j+\mu_i\pqty{1+\eps}\pqty{\bg\vdot\be}\be,
    \end{split}
\end{equation}
where $\be$ is the unit vector of the impact geometry, e.g. the unit vector pointing 
from the center of the first particle to the center of the second particle at the moment
of the collision
\begin{equation}
    \be=\frac{\br_i-\br_j}{R_i+R_j}.
\end{equation}

\subsection{Aggregation}
If the impact energy is smaller than a certain threshold 
$\ltx{E}{imp}\leqslant\ltx{E}{agg}$, the outcome of the collision is merging of two 
particles. From the momentum conservation we can write the outcome of the aggregative 
collision, which is a single particle with a mass and velocity
\begin{equation}
    m'=m_i+m_j\qquad\bv'=\bV=\frac{m_i\bv_i+m_j\bv_j}{m_i+m_j}.
\end{equation}
The total energy loss is 
\begin{equation}    
        \Delta E=\frac{MV^2}{2}-\frac{m_iv_i^2}{2}-\frac{m_jv_j^2}{2}
        =-\frac{\mu g^2}{2}=-\ltx{E}{imp},
\end{equation}
so, all the impact energy is lost during the aggregative collision.
The threshold energy value $\ltx{E}{agg}$ is in general a function of the sizes of 
particles.

\subsection{Fragmentation}
If the impact energy exceeds the certain threshold value 
$\ltx{E}{imp}\geqslant\ltx{E}{frag}$, the two impactors break into into smaller
particles in the collision.
We cannot obtain the velocities of the monomers from only conservation laws, hence 
we have to assume that certain constraints are valid. Namely, we assume two 
constraints:
\begin{enumerate}
    \item Both particles shatter into their constituent monomers;
    \item Complete isotropy of the momenta of the monomers in CoM frame;
\end{enumerate}
These two constraints allow us to write the outcome velocities of the fragmented pieces.
Let us write the energy needed to release a single monomer from a particle as $\gamma$.
Hence, the total energy needed for a complete decomposition of an aggregate of mass $m_k$
can be estimated as
\begin{equation}
    E_k = \gamma\cdot k.
\end{equation}
The fragmentation process of two particles of masses $m_i$ and $m_j$, with velocities
$\bv_i$ and $\bv_j$ can be then described as a decay of a single particle of mass 
$m_k=m_i+m_j$ with velocity $\bv_k=\bV=\mu_i\bv_i+\mu_j\bv_j$. The decay energy can 
be estimated as 
\begin{equation}
    \ltx{E}{decay}=\ltx{E}{imp}-\gamma\cdot k,
\end{equation}
which is the amount of energy which is equally distributed among all the shattered 
monomers. From this, we can see that the impact energy should be larger than 
$\gamma\cdot k$, which can be treated as the threshold energy. In the  frame, each 
released monomer has an energy 
\begin{equation}
    E'_{c}=\frac{m_1v'^2_{c}}{2}=\frac{\mu g^2}{2k}-\gamma,
\end{equation}
where $v'_{c}$ is the speed of a monomer in CoM frame
\begin{equation}
    v'_{c}=\sqrt{\frac{ij}{\pqty{i+j}^2}\cdot g^2-\frac{2\gamma}{m_1}}.
\end{equation}

Let us estimate the number of monomers $\dd{N}$ in a small solid angle $\dd{\Omega}$.
From the second constraint, we deduce that this number has to be proportional to the 
angle itself, hence 
\begin{equation}
    \dd{N}=\frac{k}{4\pi}\dd{\Omega},\qquad k=i+j.
\end{equation}

In the Lab frame, the speeds of monomers are not equal, but rather uniformly 
distribution in the range 
\begin{equation}
    \lmin{v'}=\abs{\mu_i\bv_i+\mu_j\bv_j}-
    \sqrt{\frac{ij}{i+j}\cdot g^2-\frac{2\gamma}{m_1}},\qquad
    \lmax{v'}=\abs{\mu_i\bv_i+\mu_j\bv_j}+
    \sqrt{\frac{ij}{i+j}\cdot g^2-\frac{2\gamma}{m_1}}.
\end{equation}

\section{Distribution function}




    
\end{document}