\documentclass[aps,prl,preprint,groupedaddress,10pt]{revtex4-2}
\usepackage{notations}

\begin{document}
\section{The mechanics of binary interactions}
Our system consists of $N\to\infty$ mechanically identical, spherical particles of masses 
$m_1$ and radii $R_1$. We call them monomers. These monomers can interact 
gravitationally, though their masses are very small. When monomers collide with each other,
they lose certain amount of the impact energy and rebound with a coefficient of 
restitution $\eps$. If the impact energy is less than a certain threshold value 
$\ltx{E}{imp}\leqslant\ltx{E}{agg}$, the monomers stick to each other due to surface 
forces, such as van der Waals forces, and an aggregate particle of mass $m_2$ and radius
$R_2$ appears. We call this process \emph{aggregation}. The aggregation process is a 
mechanism that creates larger particles from constituent monomers. On the other hand, 
there is also a mechanism which decreases the sizes of aggregates, which we call 
\emph{fragmentation}. If the impact energy is larger than a certain threshold value 
$\ltx{E}{imp}\geqslant\ltx{E}{frag}$, then the colliding aggregates break into smaller 
pieces. 

\subsection{Collision mechanics}
We consider a collision of two particles of masses $m_i$, $m_j$, and velocities $\bv_i$, 
$\bv_j$. If the particles did not exert gravitational influence, the collision 
geometry would have been a linear problem. However, the gravitational interaction of 
the particles result in a deflection of the trajectories of motion, aka 
\emph{gravitational focusing}. To analyze the collision mechanics of two gravitationally 
interacting bodies, let us reformulate the problem into a mathematically equivalent one.
Consider a point size particle with mass $\mu=m_im_j/M$ and velocity $\bg_0=\bv_i-\bv_j$, 
that moves in a gravitational field of a stationary body of mass $M=m_i+m_j$ and radius 
$R=R_i+R_j$. The collision happens if the point mass falls onto the surface of the 
stationary body. Let us introduce a parameter $b$, which is a distance from the center of 
the stationary mass to the trajectory asymptote of the moving particle. The vector $\bb$
is then orthogonal to the velocity $\bg$. There is a certain value of this parameter 
$\lmax{b}$, which defines the collision threshold, e.g. when the impact parameter $b$
is smaller than this threshold value, the moving particle falls onto the surface of the 
stationary body. The value of this threshold value is
\begin{equation}
    \lmax{b}\pqty{g_0}=R\sqrt{1+\frac{2GM}{Rg_0^2}},
\end{equation}
where $G$ is the gravitational constant. The value of the relative velocity $g_0$ is 
considered at infinity, hence at the point of impact, the relative speed, or in other 
words, the impact speed $g$ is 
\begin{equation}
    g = g_0\sqrt{1+\frac{2GM}{Rg_0^2}},
\end{equation}
slightly larger than $g_0$. The point of impact on the surface of the stationary body 
can be represented by a unit vector $\be$, pointing from the center of the body. However,
the actual impact speed does not depend on $\be$, i.e. as long as the impact parameter 
of the point mass particle is less than the threshold value $\lmax{b}$, 
it falls onto the surface of the stationary body with the same speed $g$ 
for any value $b$ of the impact parameter, due to energy conservation and spherical shape
of the body.

\subsection{Aggregation mechanics}




    
\end{document}