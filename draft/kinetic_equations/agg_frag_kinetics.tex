\documentclass[aps,prl,preprint,groupedaddress,10pt]{revtex4-2}
\usepackage{notations}

\begin{document}
\section{The mechanics of binary interactions}
Our system consists of $N\to\infty$ mechanically identical, spherical particles of masses 
$m_1$ and diameters $\sigma_1$. We call them monomers. These monomers can interact 
gravitationally, though their masses are very small. When monomers collide with each other,
they lose certain amount of the impact energy and rebound with a coefficient of 
restitution $\eps$. If the impact energy is less than a certain threshold value 
$\ltx{E}{imp}\leqslant\ltx{E}{agg}$, the monomers stick to each other due to surface 
forces, such as van der Waals forces, and an aggregate particle of mass $m_2$ and diameter
$\sigma_2$ appears. We call this process \emph{aggregation}. The aggregation process is a 
mechanism that creates larger particles from constituent monomers. On the other hand, 
there is also a mechanism which decreases the sizes of aggregates, which we call 
\emph{fragmentation}. If the impact energy is larger than a certain threshold value 
$\ltx{E}{imp}\geqslant\ltx{E}{frag}$, then the colliding aggregates break into smaller 
pieces. 

\subsection{Collision mechanics}
We consider a collision of two particles of masses $m_i$, $m_j$, and velocities $\bv_i$, 
$\bv_j$. If the particles did not exert gravitational influence, the collision 
geometry would have been a linear problem. However, the gravitational interaction of 
the particles result in a deflection of the trajectories of motion, aka 
\emph{gravitational stirring}.


\subsection{Aggregation mechanics}




    
\end{document}