\documentclass[aps,prl,preprint,groupedaddress,10pt]{revtex4-2}
\usepackage{notations}
\usepackage{tikz}


\begin{document}
\section{The mechanics of binary interactions}
Our system comprises an infinite number of mechanically identical, spherical particles
known as monomers. These particles have masses $m_1$ and radii $R_1$. When two
monomers collide, they lose a certain amount of impact energy and rebound with a
coefficient of restitution $\eps$. If the impact energy is below a specific threshold
value $\ltx{E}{imp}\leqslant\ltx{E}{agg}$, the monomers stick together due to surface
forces like van der Waals forces, forming a larger aggregate particle with mass $m_2$
and radius $R_2$. This process is known as \emph{aggregation} and allows the creation
of larger particles from individual monomers. A particle of mass $m_k$ is an aggregate
of $k$ monomers, hence $m_k=k\cdot m_1$. We assume that the aggregates remain
spherically shaped, and their radii scale as $R_k\sim m_k^{1/3}\sim k^{1/3}$.

Conversely, there is another mechanism called \emph{fragmentation} that decreases the
sizes of aggregates. If the impact energy is higher than a certain threshold value
$\ltx{E}{imp}\geqslant\ltx{E}{frag}$, the colliding aggregates break into smaller pieces.
The size distribution of the fragmented pieces are difficult to model analytically, and
in this work we assume a simplistic model of fragmentation, called \emph{shattering}.
When two aggregates of masses $m_i$ and $m_j$ collide with a sufficient energy, both
of them shatter into singular monomers, $m_i\to i\cdot m_1$ and $m_j\to j\cdot m_1$.

\subsection{Generalized collisions}
Let us consider a collision of particles of masses $m_i$, $m_j$ and
velocities $\bv_i$, $\bv_j$, and radii $R_i$, $R,j$. The collision geometry is
characterized by the unit vector $\bn$, which is directed from the center of particle
$j$ to the center of particle $i$ at the moment of contact of two particles
\begin{equation}
    \bn=\frac{\br_i-\br_j}{R_i+R_j},
\end{equation}
where $\br_i$ and $\br_j$ are position vectors of the particles. The next parameter
which describes the collision, is the \emph{restitution coefficient}
$0\leqslant\eps\leqslant 1$. This parameter controls the amount of energy dissipated
after the collision. The total energy of this binary system, can be split into two parts,
the translational energy and the internal energy
\begin{equation}
    E=\ltx{E}{translation}+\ltx{E}{internal}=\frac{MV^2}{2}+\frac{\mu g^2}{2},
\end{equation}
where
\begin{equation}
    \begin{split}
        &\bV=\mu_i\bv_i+\mu_j\bv_j,\quad\bg=\bv_i-\bv_j,\\
        &M=m_i+m_j,\quad\mu=\frac{m_im_j}{m_i+m_j},\\
        &\mu_i=\frac{m_i}{m_i+m_j},\quad\mu_j=\frac{m_j}{m_i+m_j}.
    \end{split}
\end{equation}
The translational energy does not change after the collision, but the internal part
dissipates. Using $\bn$, we can split the relative velocity into normal and tangential
parts
\begin{equation}
    \bg_n=\pqty{\bg\vdot\bn}\bn,\quad\bg_t=\bg-\bg_n,
\end{equation}
and write the total energy as
\begin{equation}
    E=\frac{MV^2}{2}+\frac{\mu g_t^2}{2}+\frac{\mu g_n^2}{2}.
\end{equation}
Now, we can write the post-collision total energy $E'$ as
\begin{equation}
    E'=\frac{MV^2}{2}+\frac{\mu g_t^2}{2}+\eps^2\frac{\mu g_n^2}{2},
\end{equation}
where only the normal part of the internal energy dissipates.

In the most general case, we assume that
the outcome of the collision is a collection of particles with various masses and
velocities. Introducing the function $P_k\pqty{\bv_k\vert\bv_i,\bv_j}$, which is the number of
particles of mass $m_k$ and velocity $\bv_k$ created after the collision of particles with
velocities $\bv_i$ and $\bv_j$ and sizes $i,j$,
or in other words, introducing the velocity distribution function of the particles of
mass $m_k$. Using this distribution function, we can write the total mass, momentum and
energy of particles in the outcome of the generalized collision
\begin{equation}
    \begin{split}
        M &= \sum_{k=1}^{i+j}\int\dd{\bv}m_kP_k\pqty{\bv_k\vert\bv_i,\bv_j},\\
        M\bV &= \sum_{k=1}^{i+j}\int\dd{\bv}m_k\bv P_k\pqty{\bv_k\vert\bv_i,\bv_j},\\
        \frac{MV^2}{2}+\frac{\mu g_t^2}{2}+\eps^2\frac{\mu g_n^2}{2} &=
        \sum_{k=1}^{i+j}\int\dd{\bv}\frac{m_kv^2}{2}P_k\pqty{\bv_k\vert\bv_i,\bv_j}.
    \end{split}
\end{equation}

\subsection{Restitution}
For a restitutive rebound of particles, the outcome velocities are analytic and given by
\begin{equation}
    \begin{split}
        \bv_i'&=\bv_i-\mu_j\pqty{1+\eps}\bg_n,\\
        \bv_j'&=\bv_j+\mu_i\pqty{1+\eps}\bg_n,
    \end{split}
\end{equation}
Let us write the distribution function $P_k\pqty{\bv}$ for the restitutive collision.
First of all, the masses of impacting particles do not change, hence the distribution
function should contain $\delta$-functions to control this. Together with the analytic
expression for the outcome velocities, we can write
\begin{equation}
    \utx{P_k}{res}\pqty{\bv_k\vert\bv_i,\bv_j}=
    \delta_{k,i}\delta\bqty{\bv_k-\bv_i+\mu_j\pqty{1+\eps}\bg_n}+
    \delta_{k,j}\delta\bqty{\bv_k-\bv_j-\mu_i\pqty{1+\eps}\bg_n}.
\end{equation}
The $\delta_{k,x}$ is a Kronecker-delta operator.

\subsection{Aggregation}
If the impact energy is smaller than a certain threshold
$\ltx{E}{imp}\leqslant\ltx{E}{agg}$, the outcome of the collision is merging of two
particles. From the momentum conservation we can write the outcome of the aggregative
collision, which is a single particle with a mass and velocity
\begin{equation}
    m'=m_i+m_j\qquad\bv'=\bV=\frac{m_i\bv_i+m_j\bv_j}{m_i+m_j}.
\end{equation}
The total energy loss is
\begin{equation}
    \Delta E=\frac{MV^2}{2}-\frac{m_iv_i^2}{2}-\frac{m_jv_j^2}{2}
    =-\frac{\mu g^2}{2}=-\ltx{E}{internal},
\end{equation}
so, all the internal energy is lost during the aggregative collision.
The threshold energy value $\ltx{E}{agg}$ is in general a function of the sizes of
particles.

Let us write the debris velocity distribution function for the aggregation process.
Since the outcome is a single particle of mass $m_i+m_j$, with velocity $\bV$, we have
\begin{equation}
    \utx{P_k}{agg}\pqty{\bv_k\vert\bv_i,\bv_j}=\delta_{k,i+j}
    \delta\bqty{\bv_k-\mu_i\bv_i-\mu_j\bv_j}.
\end{equation}


\subsection{Fragmentation}
If the impact energy exceeds the certain threshold value
$\ltx{E}{imp}\geqslant\ltx{E}{frag}$, the two impactors break into into smaller
particles in the collision.
We cannot obtain the velocities of the monomers from only conservation laws, hence
we have to assume that certain constraints are valid. Namely, we assume two
constraints:
\begin{enumerate}
    \item Both particles shatter into their constituent monomers;
    \item Complete isotropy of the momenta of the monomers in CoM frame;
\end{enumerate}
These two constraints allow us to write the outcome velocities of the fragmented pieces.
Let us write the energy needed to release a single monomer from a particle as $\gamma$.
Hence, the total energy needed for a complete decomposition of an aggregate of mass $m_k$
can be estimated as
\begin{equation}
    E_k = \gamma\cdot k.
\end{equation}
The fragmentation process of two particles of masses $m_i$ and $m_j$, with velocities
$\bv_i$ and $\bv_j$ can be then described as a decay of a single particle of mass
$m_k=m_i+m_j$ with velocity $\bv_k=\bV=\mu_i\bv_i+\mu_j\bv_j$. The decay energy can
be estimated as
\begin{equation}
    \ltx{E}{decay}=\ltx{E}{imp}-\gamma\cdot k,
\end{equation}
which is the amount of energy which is equally distributed among all the shattered
monomers. From this, we can see that the impact energy should be larger than
$\gamma\cdot k$, which can be treated as the threshold energy. Since the impact energy
is the normal part of the internal energy, we can write
\begin{equation}
    \ltx{E}{decay}=\frac{\mu g_n^2}{2}-\gamma\cdot k=\eps^2\frac{\mu g_n^2}{2},
\end{equation}
and the restitution coefficient for the fragmentation is
\begin{equation}
    \eps=\sqrt{1-\frac{2\gamma k}{\mu g_n^2}}.
\end{equation}
Since the decay energy has to be positive, we can write the threshold value for the
normal relative velocity as
\begin{equation}
    g_n\geqslant\sqrt{\frac{2\gamma k}{\mu}}=
    \sqrt{\frac{2\gamma}{m_1}}\cdot\frac{i+j}{\sqrt{ij}}.
\end{equation}
In the CoM frame, each released monomer has an energy
\begin{equation}
    E'_{c}=\frac{m_1v'^2_{c}}{2}=\frac{\ltx{E}{decay}}{k}=\eps^2\frac{\mu g_n^2}{2k},
\end{equation}
where $v'_{c}$ is the speed of a monomer in CoM frame
\begin{equation}
    v'_{c}=\frac{\sqrt{ij}}{i+j}\cdot\eps g_n.
\end{equation}

Let us estimate the number of monomers $\dd{N}$ in a small solid angle $\dd{\Omega}$.
From the second constraint, we deduce that this number has to be proportional to the
angle itself, hence
\begin{equation}
    \dd{N}=\frac{k}{4\pi}\dd{\Omega},\qquad k=i+j.
\end{equation}

In the Lab frame, the speeds of monomers are not equal, but rather uniformly
distribution in the range
\begin{equation}
    \lmin{v'}=V-v'_c,\qquad\lmax{v'}=V+v'_c.
\end{equation}

Since the fragmented debris consist of only monomers, the distribution function
$P_k\pqty{\bv_k\vert\bv_i,\bv_j}$ has to contain the term $\delta_{k,1}$.
In the CoM frame, we can write
\begin{equation}
    \utx{P_k}{frag, CoM}\pqty{\bv_k\vert\bv_i,\bv_j}=
    \delta_{k,1}\delta\pqty{v_k-v'_c}\frac{i+j}{4\pi}.
\end{equation}
In this case, the integral of any velocity function $\varphi\pqty{\bv_k}$ in the form of
\begin{equation}
    \int\dd{\bv_k}\varphi\pqty{\bv_k}\utx{P_k}{frag,CoM}\pqty{\bv_k\vert\bv_i,\bv_j}=
    \delta_{k,1}\frac{i+j}{4\pi}\int\dd{\bv_k}\varphi\pqty{\bv_k}\delta\pqty{v_k-v'_c},
\end{equation}
can be written as
\begin{equation}
    \delta_{k,1}\frac{i+j}{4\pi}\int\dd{\be}
    \int_{0}^{\infty}\dd{v}\varphi\pqty{v,\be}\delta\pqty{v-v'_c}=
    \delta_{k,1}\frac{i+j}{4\pi}\int\dd{\be}\varphi\pqty{v'_c,\be}.
\end{equation}
If $\varphi\pqty{\bv}\equiv\varphi\pqty{v,\be}=\be\varphi\pqty{v}$, such as $\bv=v\be$,
then
\begin{equation}
    \int\dd{\be}\be\varphi\pqty{v}=\bnull.
\end{equation}
If $\varphi\pqty{\bv}\equiv\varphi\pqty{v,\be}=\varphi\pqty{v}$, then
\begin{equation}
    \int\dd{\be}\varphi\pqty{v}=4\pi\varphi\pqty{v}.
\end{equation}

To write the debris velocity distribution function in the Lab frame, we have to
add the center of mass velocity to all the velocities of the monomers. This can be
written as
\begin{equation}
    \utx{P_k}{frag}\pqty{\bv_k\vert\bv_i,\bv_j}=
    \delta_{k,1}\frac{i+j}{4\pi}\int\dd{\be}\delta\pqty{\bv_k-\bV-v'_c\be}.
\end{equation}
Now, integrating over a function $\varphi\pqty{\bv_k}$ becomes
\begin{equation}
    \delta_{k,1}\frac{i+j}{4\pi}\int\dd{\be}\int\dd{\bv_k}\varphi\pqty{\bv_k}
    \delta\pqty{\bv_k-\bV-v'_c\be}=
    \delta_{k,1}\frac{i+j}{4\pi}\int\dd{\be}\varphi\pqty{\bV-v'_c\be}.
\end{equation}
If $\varphi\pqty{\bV-v'_c\be}=\varphi\pqty{\bV}-\be\varphi\pqty{v'_c}$, then
we have
\begin{equation}
    \delta_{k,1}\frac{i+j}{4\pi}\int\dd{\be}\int\dd{\bv}\varphi\pqty{\bv}
    \delta\pqty{\bv-\bV-v'_c\be}=\delta_{k,1}\pqty{i+j}\varphi\pqty{\bV}.
\end{equation}


\section{Distribution function}
The statistical description of the system is fully described by a set of distribution
functions $f_k\pqty{\br,\bv,t}$. It is normalized, such that
$f_k\pqty{\br,\bv,t}\dd{\br}\dd{\bv}$ gives the number of particles of size $k$
in the phase space volume $\dd{\Gamma}=\dd{\br}\dd{\bv}$, around the point
$\pqty{\br,\bv}$. Hence, integrating over the whole phase space gives us the total
number of particles of size $k$
\begin{equation}
    N_k=\int\dd{\br}\dd{\bv}f_k\pqty{\br,\bv,t}.
\end{equation}
The spacial distribution of particles is not very important for us, hence in the following
we assume that the system is spatially homogeneous, and we use only the velocity
distribution function $f_k\pqty{\bv,t}$
\begin{equation}
    N_k=\int\dd{\br}\int\dd{\bv}f_k\pqty{\bv,t},
\end{equation}
hence
\begin{equation}
    n_k\equiv\frac{N_k}{V}=\int\dd{\bv}f_k\pqty{\bv,t},
\end{equation}
is the number density of the subsystem of particles with size $k$. The other field
functions, such as the mean flow velocity $\bu_k$ or granular temperature $T_k$
can be defined as velocity moments of the distribution function
\begin{equation}
    \begin{split}
        n_k\bu_k &= \int\dd{\bv}\bv f_k\pqty{\bv,t},\\
        \frac{3}{2}n_kT_k &=\int\dd{\bv}\frac{m_kc_k^2}{2}f_k\pqty{\bv,t},\\
        \bc_k &= \bv-\bu_k.
    \end{split}
\end{equation}

\section{Kinetic equations}
The time evolution of the distribution functions obey the Boltzmann equations
\begin{equation}\label{eq:kinetic_equation_generic}
    \pqty{\pdv{t}+\bv\vdot\pdv{\br}-\frac{1}{m_k}\pdv{U(r)}{\br}\vdot\pdv{\bv}}
    f_k\pqty{\br,\bv,t}=\mathcal{I}_k\pqty{f_i,f_j},
\end{equation}
where $U(r)$ is the potential of the external gravitational field. The LHS
of the Boltzmann equation describes the change over time in the function $f_k$ due to the
local flow of the particles, subject to external driving. The function
$\mathcal{I}\pqty{f_i,f_j}$ on the RHS is the \emph{collision integral}, which
describes the change over time in the function $f_k$ due to collisions of particles $i$
with particles of size $j$. Since we have three types of collisional outcomes, the
collision integral $\mathcal{I}$ has to take into account all these types of outcomes.
Without the loss of generality, we can write the collision integral as a sum of three
functions
\begin{equation}
    \mathcal{I}_k\pqty{f_i,f_j}=\utx{\mathcal{I}_k}{agg}\pqty{f_i,f_j}+
    \utx{\mathcal{I}_k}{res}\pqty{f_i,f_j}+\utx{\mathcal{I}_k}{frag}\pqty{f_i,f_j},
\end{equation}
each corresponding to the specific type of collision.

\subsection{General structure of collision integrals}
Let us consider a collision integral $\mathcal{J}\pqty{f_i,f_j}$ for a generalized
collision. If we consider a small volume in the phase space $\dd{\Gamma}$
around a point $\pqty{\br,\bv}$, the term $f_i\pqty{\br,\bv,t}\dd{\Gamma}$ gives us
the number of particles of size $i$ in that volume at time $t$. The collision integral
shows how many particles leave and enter this phase space volume per unit time, due
to collisions only. So, the collision integral contains two terms, the gain term, which
shows the number of particles that enter the phase space volume per unit time, and the
loss terms, which shows the number of particles that leave this phase space volume per
unit time
\begin{equation}
    \mathcal{J}_k\pqty{f_i,f_j}=\mathcal{G}_k\pqty{f_i,f_j}-\mathcal{L}_k\pqty{f_i,f_j}.
\end{equation}

These terms are proportional to the number of collisions happening in unit volume per unit
time. Estimation of the number of collisions between the particles of sizes $i$ and $j$
gives us
\begin{equation}
    \utx{\dd{N}}{cols}_{ij}=\sigma_{ij}^2\dd{\bv_i}\dd{\bv_j}\int\dd{\bn}
    \Theta\pqty{-\bg\vdot\bn}\abs{\bg\vdot\bn}f_i\pqty{\bv_i,t}f_j\pqty{\bv_j,t}.
\end{equation}
Now, integrating over all possible pairs of velocities and mass combinations, we can write
the gain and loss terms as
\begin{equation}
    \begin{split}
        \mathcal{G}_k\pqty{f_i,f_j}&=\sum_{i,j}\sigma_{ij}^2\int\dd{\bv_i}\dd{\bv_j}
        \int\dd{\Omega}P_k\pqty{\bv_k\vert\bv_i,\bv_j}f_i\pqty{\bv_i,t}f_j\pqty{\bv_j,t},\\
        \mathcal{L}_k\pqty{f_i,f_j}&=\sum_{i,j}\sigma_{ij}^2\int\dd{\bv_i}\dd{\bv_j}
        \int\dd{\Omega}\delta_{k,i}\delta\pqty{\bv_i-\bv_k}f_i\pqty{\bv_i,t}f_j\pqty{\bv_j,t}.
    \end{split}
\end{equation}
The delta functions in the loss term make sure that one of the collision partners is always
the considered particle of size $k$ and velocity $\bv_k$. Now, the collision integral for a
general type of collision is written as
\begin{equation}
    \mathcal{I}_k\pqty{f_i,f_j}=\sum_{i,j}\sigma_{ij}^2\int\dd{\bv_i}\dd{\bv_j}\int\dd{\Omega}
    \bqty{P_k\pqty{\bv_k\vert\bv_i,\bv_j}-\delta_{k,i}\delta\pqty{\bv_i-\bv_k}}f_if_j,
\end{equation}
where
\begin{equation}
    \dd{\Omega}=\dd{\bn}\Theta\pqty{-\bg\vdot\bn}\abs{\bg\vdot\bn},\quad\bg=\bv_i-\bv_j.
\end{equation}

For specific types of collisions, e.g. aggregation, fragmentation, restitution, we have
to make sure that the integration domains are specified as well. Usually, this domains are
the function of the relative velocity $\mathcal{D}(\bg)$. Hence, the specific collision
integrals are written as
\begin{equation}
    \utx{\mathcal{I}_k}{type}\pqty{f_i,f_j}=\sum_{i,j}\sigma_{ij}^2\int\dd{\bv_i}\dd{\bv_j}
    \utx{\mathcal{D}}{type}(\bg)\int\dd{\Omega}\bqty{\utx{P_k}{type}
        \pqty{\bv_k\vert\bv_i,\bv_j}-\delta_{k,i}\delta\pqty{\bv_i-\bv_k}}f_if_j,
\end{equation}
where \emph{type} can be aggregation, fragmentation or restitution.

\section{Hydrodynamic equations}
Using the kinetic equations, we can construct the balance equations for macroscopic or
hydrodynamic fields. Namely, we focus on three of them, the number density fields
$\Bqty{n_k\pqty{\br,t}}$, the mean velocity field $\bu_k=\bu\pqty{\br,t}$, and the temperature
fields $\Bqty{T_k\pqty{\br,t}}$. All these fields can be defined as certain moments of the
distribution function $f_k\pqty{\bv, \br,t}$
\begin{equation}
    \begin{split}
        n_k\pqty{\br,t} &= \int\dd{\bv} f_k\pqty{\bv,\br,t},\\
        n_k\bu_k\pqty{\br,t} &= \int\dd{\bv} \bv f_k\pqty{\bv,\br,t},\\
        \frac{3}{2}n_kT_k\pqty{\br,t} &= \int\dd{\bv}\frac{m_kc_k^2}{2}f_k\pqty{\bv,\br,t},\\
        \bc_k\pqty{\br,t} &= \bv - \bu_k\pqty{\br,t}.
    \end{split}
\end{equation}
Also, we introduce the momentum flux (pressure tensor) and heat flux terms as
\begin{equation}
    \begin{split}
        \bP_k\pqty{\br,t} &= \int\dd{\bv}m_k\bc_k\bc_kf_k\pqty{\br,\bv,t},\\
        \bq_k\pqty{\br,t} &= \int\dd{\bv}\frac{m_kc_k^2}{2}\bc_kf_k\pqty{\br,\bv,t}.
    \end{split}
\end{equation}
Here, the notation with two vectors next to each other $\bc\bc$ denote a dyadic tensor, while
$\bc\vdot\bc$ is a dot product.

We can write the mean field equations, by averaging the parameters over all ensembles $k$.
\begin{equation}
    \begin{split}
        n\pqty{\br,t} &= \sum_kn_k\pqty{\br,t}=const,\\
        n\bu\pqty{\br,t} &= \sum_kn_k\bu_k\pqty{\br,t},\\
        nT\pqty{\br,t} &= \sum_kn_kT_k\pqty{\br,t}.
    \end{split}
\end{equation}

Since the total mass of the entire system is constant $n\pqty{\br,t}=const$, we need a more
informative parameter to describe the dynamics of the mean mass of the system. We can use the
squared number density as a more informative parameter
\begin{equation}
    \bar{n}\pqty{\br,t}=\frac{1}{n\pqty{\br,t}}\sum_kn_k^2\pqty{\br,t}.
\end{equation}
In the following, we normalize the total number density as one $n\pqty{\br,t}=1$ for the sake
of brevity. The hydrodynamic balance equations can be obtained by multiplying the kinetic
equation with specific functions and integrating over the velocities.

\subsection{Balance equations}
Let us take a certain function of the velocity $\psi_k(\bv)$, which describe a specific physical
characteristics of the system. We can multiply the kinetic equation (\ref{eq:kinetic_equation_generic})
by this function, and integrate over the velocity
\begin{equation}
    \int\dd{\bv}\pqty{\pdv{t}+\bv\vdot\pdv{\br}+\bw\vdot\pdv{\bv}}
    \psi_k(\bv)f_k\pqty{\br,\bv,t}=\int\dd{\bv}\mathcal{I}_k\pqty{f_i,f_j}\psi_k(\bv),
\end{equation}
where
\begin{equation}
    \bw = -\frac{1}{m_k}\pdv{U(r)}{\br}.
\end{equation}
The RHS of this equation can be written as a difference of gain and loss terms, without a loss
of generality
\begin{equation}
    \int\dd{\bv}\mathcal{I}_k\pqty{f_i,f_j}\psi_k(\bv)=\mathcal{G}(\psi_k)-\mathcal{L}(\psi_k)=
    \ltx{\avg{\pdv{t}\pqty{n_k\psi_k}}}{coll},
\end{equation}
where the exact forms of the gain and loss functions are obtained only after specifying the
distribution function $f_k\pqty{\br,\bv,t}$. If the physical characteristic specified by the function
$\psi_k(\bv)$ is conserved after a collision, the gain and loss terms are identical and the RHS
vanish. The piece by piece integration gives us the next generalized form for balance equations
of the physical value $\psi_k(\bv)$
\begin{equation}
    \pdv{t}\int\dd{\bv}\psi_k\pqty{\ib{v}}f_k+
    \pdv{\ia{r}}\int\dd{\bv}\psi_k\pqty{\ib{v}}\ia{v}f_k-
    \ib{w}\int\dd{\bv}\pdv{\psi_k\pqty{\ia{v}}}{\ia{v}}f_k=
    \ltx{\avg{\pdv{t}\pqty{n_k\psi_k}}}{coll},
\end{equation}
where we use the index notation forms, and greek letters $\alpha,\beta,\gamma$ denote the
coordinate indices. Also, we invoke the summation notation convention for the repeated indices.
By plugging in the functions $\psi_k=m_k$, $\psi_k=m_k\ib{v}$ and $\psi_k=m_kv^2/2$, we obtain
the balance equations for the mass density, momentum density and energy density
\begin{equation}
    \begin{split}
        \pdv{\rho_k}{t}+\pdv{\ia{r}}\pqty{\rho_k\ia{u}} &=\ltx{\avg{\pdv{\rho_k}{t}}}{coll},\\
        \rho_k\pdv{\ib{u}}{t}+\rho_k\ia{u}\pdv{\ib{u}}{\ia{r}}+\pdv{\iab{P}}{\ia{r}}&=
        \rho_k\ib{w}-\ib{u}\ltx{\avg{\pdv{\rho_k}{t}}}{coll},\\
        \pdv{t}\pqty{\frac{3}{2}n_kT_k}+\pdv{\ia{r}}\pqty{\frac{3}{2}n_kT_k\ib{u}}+
        \pdv{\ia{q}}{\ia{r}}+\iab{P}\pdv{\ib{u}}{\ia{r}}&=
        \ltx{\avg{\pdv{t}\pqty{\frac{3}{2}n_kT_k}}}{coll}+
        \frac{u^2}{2}\ltx{\avg{\pdv{\rho_k}{t}}}{coll}.
    \end{split}
\end{equation}

To proceed further, we assume certain condition in our system. First condition is that the
mass density and temperature fields are homogeneous in space, hence their gradients vanish.
This also implies that the heat flux is absent. Second condition is that the mean velocity
field is stationary but has a steady spacial gradient. These conditions give us
\begin{equation}
    \pdv{\rho_k}{\ia{r}}=\bnull,\quad\pdv{T_k}{\ia{r}}=\bnull,\quad\ia{q}=\bnull.
    \quad\pdv{\ia{u}}{t}=0,
\end{equation}

Using these conditions, our balance equations simplify into
\begin{equation}
    \begin{split}
        \pdv{\rho_k}{t}+\rho_k\pdv{\ia{u}}{\ia{r}} &=\ltx{\avg{\pdv{\rho_k}{t}}}{coll},\\
        \rho_k\ia{u}\pdv{\ib{u}}{\ia{r}}+\pdv{\iab{P}}{\ia{r}}&=
        \rho_k\ib{w}-\ib{u}\ltx{\avg{\pdv{\rho_k}{t}}}{coll},\\
        \pdv{t}\pqty{\frac{3}{2}n_kT_k}+\frac{3}{2}n_kT_k\pdv{\ia{u}}{\ia{r}}+
        \iab{P}\pdv{\ib{u}}{\ia{r}}&=
        \ltx{\avg{\pdv{t}\pqty{\frac{3}{2}n_kT_k}}}{coll}+
        \frac{u^2}{2}\ltx{\avg{\pdv{\rho_k}{t}}}{coll}.
    \end{split}
\end{equation}

Next, we can assume a zero divergence mean velocity field, which yields the balance
equations in the next form
\begin{equation}
    \begin{split}
        \pdv{\rho_k}{t} &= \ltx{\avg{\pdv{\rho_k}{t}}}{coll},\\
        \pdv{t}\pqty{\frac{3}{2}n_kT_k} &= -\iab{P}\pdv{\ib{u}}{\ia{r}}+
        \ltx{\avg{\pdv{t}\pqty{\frac{3}{2}n_kT_k}}}{coll}+
        \frac{u^2}{2}\ltx{\avg{\pdv{\rho_k}{t}}}{coll}.
    \end{split}
\end{equation}
To close the balance equations, we have to write the pressure tensor in terms of the macroscopic
fields $\rho_k$, $\bu$ and $T_k$. In the simplest form, we can write the empirical dependence
\begin{equation}
    \iab{P} = p\iab{\delta}-\eta\pqty{\pdv{\ia{u}}{\ib{r}}+\pdv{\ib{u}}{\ia{r}}
        -\frac{2}{3}\iab{\delta}\pdv{\ig{u}}{\ig{r}}},
\end{equation}
where $p$ is the hydrostatic pressure and $\eta$ is the shear viscosity coefficient. It is shown
that the hydrostatic pressure depends on the granular temperature through the equation of state
\begin{equation}
    p = nT\pqty{1+\frac{1+\eps}{3}\pi n\sigma^{3}g_2\pqty{\sigma}}.
\end{equation}
Writing a similar form for te polydisperse system, and assuming the zero divergence mean velocity
field, we have for the pressure tensor
\begin{equation}
    \iab{P}=n_kT_k(1+\phi_k)\iab{\delta}-\eta\pqty{\pdv{\ia{u}}{\ib{r}}+\pdv{\ib{u}}{\ia{r}}},
\end{equation}
and write the final form of the balance equations as
\begin{equation}
    \begin{split}
        \pdv{n_k}{t} &= \ltx{\avg{\pdv{n_k}{t}}}{coll},\\
        \pdv{t}\pqty{\frac{3}{2}n_kT_k} &=
        \eta_k\bqty{\pdv{\ia{u}}{\ib{r}}\pdv{\ib{u}}{\ia{r}}+\pdv{\ib{u}}{\ia{r}}\pdv{\ib{u}}{\ia{r}}}+
        \ltx{\avg{\pdv{t}\pqty{\frac{3}{2}n_kT_k}}}{coll}+
        \frac{m_ku^2}{2}\ltx{\avg{\pdv{n_k}{t}}}{coll}.
    \end{split}
\end{equation}
Here we used
\begin{equation}
    \iab{\delta}\pdv{\ib{u}}{\ia{r}}=\pdv{\ia{u}}{\ia{r}}=0.
\end{equation}
Since
\begin{equation}
    \pdv{t}\pqty{\frac{3}{2}n_kT_k}=\frac{3}{2}n_k\pdv{T_k}{t}+
    \frac{3}{2}T_k\pdv{n_k}{t}=\frac{3}{2}n_k\pdv{T_k}{t}+
    \frac{3}{2}T_k\ltx{\avg{\pdv{n_k}{t}}}{coll},
\end{equation}
we can rewrite the hydrodynamic balance equations as 
\begin{equation}
    \begin{split}
        \pdv{n_k}{t} &= \ltx{\avg{\pdv{n_k}{t}}}{coll},\\
        \pdv{T_k}{t} &=
        \frac{2\eta_k}{3n_k}\bqty{\pdv{\ia{u}}{\ib{r}}\pdv{\ib{u}}{\ia{r}}+
        \pdv{\ib{u}}{\ia{r}}\pdv{\ib{u}}{\ia{r}}}+
        \frac{2}{3n_k}\ltx{\avg{\pdv{t}\pqty{\frac{3}{2}n_kT_k}}}{coll}+
        \frac{2}{3n_k}\pqty{\frac{m_ku^2}{2}-\frac{3T_k}{2}}\ltx{\avg{\pdv{n_k}{t}}}{coll}.
    \end{split}
\end{equation}

\section{Calculating collision integrals with Maxwellian distribution functions}
Now we calculate the collisional terms for the number density and temperature 
evolutions assuming the Maxwellian distribution function
\begin{equation}
    f_k\pqty{\bv,t}=\pqty{\frac{n_k}{2\pi T_k}}^{3/2}\cdot
    \exp\pqty{-\frac{m_kv^2}{2T_k}},
\end{equation}
where the time dependence of the distribution function comes from the evolution
of the parameters $n_k(t)$ and $T_k(t)$.

Each of the collision terms are obtained by solving the next integrals
\begin{equation}
    \ltx{\avg{\pdv{n_k}{t}}}{coll} = \int\dd{\bv_k}\utx{\mathcal{I}}{agg}(\bv_k)+
    \int\dd{\bv_k}\utx{\mathcal{I}}{res}(\bv_k)+
    \int\dd{\bv_k}\utx{\mathcal{I}}{frag}(\bv_k),
\end{equation}
and
\begin{equation}
    \ltx{\avg{\pdv{t}\pqty{\frac{3}{2}n_kT_k}}}{coll} = 
    \int\dd{\bv_k}\frac{m_kv_k^2}{2}\utx{\mathcal{I}}{agg}(\bv_k)+
    \int\dd{\bv_k}\frac{m_kv_k^2}{2}\utx{\mathcal{I}}{res}(\bv_k)+
    \int\dd{\bv_k}\frac{m_kv_k^2}{2}\utx{\mathcal{I}}{frag}(\bv_k).
\end{equation}














\end{document}