\documentclass[aps,prl,preprint,groupedaddress,10pt]{revtex4-2}
\usepackage{notations}
\usepackage{tikz}


\begin{document}
\section{The mechanics of binary interactions}
Our system comprises an infinite number of mechanically identical, spherical particles 
known as monomers. These particles have masses $m_1$ and radii $R_1$. When two 
monomers collide, they lose a certain amount of impact energy and rebound with a 
coefficient of restitution $\eps$. If the impact energy is below a specific threshold 
value $\ltx{E}{imp}\leqslant\ltx{E}{agg}$, the monomers stick together due to surface 
forces like van der Waals forces, forming a larger aggregate particle with mass $m_2$ 
and radius $R_2$. This process is known as \emph{aggregation} and allows the creation 
of larger particles from individual monomers. A particle of mass $m_k$ is an aggregate
of $k$ monomers, hence $m_k=k\cdot m_1$. We assume that the aggregates remain 
spherically shaped, and their radii scale as $R_k\sim m_k^{1/3}\sim k^{1/3}$.

Conversely, there is another mechanism called \emph{fragmentation} that decreases the 
sizes of aggregates. If the impact energy is higher than a certain threshold value 
$\ltx{E}{imp}\geqslant\ltx{E}{frag}$, the colliding aggregates break into smaller pieces.
The size distribution of the fragmented pieces are difficult to model analytically, and
in this work we assume a simplistic model of fragmentation, called \emph{shattering}.
When two aggregates of masses $m_i$ and $m_j$ collide with a sufficient energy, both 
of them shatter into singular monomers, $m_i\to i\cdot m_1$ and $m_j\to j\cdot m_1$.

\subsection{Generalized collisions}
Let us consider a collision of particles of masses $m_i$, $m_j$ and 
velocities $\bv_i$, $\bv_j$, and radii $R_i$, $R,j$. The collision geometry is 
characterized by the unit vector $\bn$, which is directed from the center of particle 
$j$ to the center of particle $i$ at the moment of contact of two particles
\begin{equation}
    \bn=\frac{\br_i-\br_j}{R_i+R_j},
\end{equation}
where $\br_i$ and $\br_j$ are position vectors of the particles. The next parameter 
which describes the collision, is the \emph{restitution coefficient} 
$0\leqslant\eps\leqslant 1$. This parameter controls the amount of energy dissipated 
after the collision. The total energy of this binary system, can be split into two parts, 
the translational energy and the internal energy
\begin{equation}
    E=\ltx{E}{translation}+\ltx{E}{internal}=\frac{MV^2}{2}+\frac{\mu g^2}{2},
\end{equation}
where 
\begin{equation}
    \begin{split}
        &\bV=\mu_i\bv_i+\mu_j\bv_j,\quad\bg=\bv_i-\bv_j,\\
        &M=m_i+m_j,\quad\mu=\frac{m_im_j}{m_i+m_j},\\
        &\mu_i=\frac{m_i}{m_i+m_j},\quad\mu_j=\frac{m_j}{m_i+m_j}.
    \end{split}
\end{equation}
The translational energy does not change after the collision, but the internal part 
dissipates. Using $\bn$, we can split the relative velocity into normal and tangential
parts
\begin{equation}
    \bg_n=\pqty{\bg\vdot\bn}\bn,\quad\bg_t=\bg-\bg_n,
\end{equation}
and write the total energy as 
\begin{equation}
    E=\frac{MV^2}{2}+\frac{\mu g_t^2}{2}+\frac{\mu g_n^2}{2}.
\end{equation}
Now, we can write the post-collision total energy $E'$ as 
\begin{equation}
    E'=\frac{MV^2}{2}+\frac{\mu g_t^2}{2}+\eps^2\frac{\mu g_n^2}{2},
\end{equation}
where only the normal part of the internal energy dissipates.

In the most general case, we assume that
the outcome of the collision is a collection of particles with various masses and 
velocities. Introducing the function $P_k\pqty{\bv}$, which is the number of 
particles of mass $m_k$ and velocity $\bv$ created after the collision, 
or in other words, introducing the velocity distribution function of the particles of 
mass $m_k$. Using this distribution function, we can write the total mass, momentum and 
energy of particles in the outcome of the generalized collision
\begin{equation}
    \begin{split}
        M &= \sum_{k=1}^{i+j}\int\dd{\bv}m_kP_k\pqty{\bv},\\
        M\bV &= \sum_{k=1}^{i+j}\int\dd{\bv}m_k\bv P_k\pqty{\bv},\\
        \frac{MV^2}{2}+\frac{\mu g_t^2}{2}+\eps^2\frac{\mu g_n^2}{2} &= 
        \sum_{k=1}^{i+j}\int\dd{\bv}\frac{m_kv^2}{2}P_k\pqty{\bv}.
    \end{split}
\end{equation}

\subsection{Restitution}
For a restitutive rebound of particles, the outcome velocities are analytic and given by
\begin{equation}
    \begin{split}
        \bv_i'&=\bv_i-\mu_j\pqty{1+\eps}\bg_n,\\
        \bv_j'&=\bv_j+\mu_i\pqty{1+\eps}\bg_n,
    \end{split}
\end{equation}
Let us write the distribution function $P_k\pqty{\bv}$ for the restitutive collision.
First of all, the masses of impacting particles do not change, hence the distribution
function should contain $\delta$-functions to control this. Together with the analytic 
expression for the outcome velocities, we can write
\begin{equation}
    \utx{P_k}{res}\pqty{\bv}=
    \delta_{k,i}\delta\bqty{\bv-\bv_i+\mu_j\pqty{1+\eps}\bg_n}+
    \delta_{k,j}\delta\bqty{\bv-\bv_j-\mu_i\pqty{1+\eps}\bg_n}.
\end{equation}
The $\delta_{k,x}$ is a Kronecker-delta operator.

\subsection{Aggregation}
If the impact energy is smaller than a certain threshold 
$\ltx{E}{imp}\leqslant\ltx{E}{agg}$, the outcome of the collision is merging of two 
particles. From the momentum conservation we can write the outcome of the aggregative 
collision, which is a single particle with a mass and velocity
\begin{equation}
    m'=m_i+m_j\qquad\bv'=\bV=\frac{m_i\bv_i+m_j\bv_j}{m_i+m_j}.
\end{equation}
The total energy loss is 
\begin{equation}    
        \Delta E=\frac{MV^2}{2}-\frac{m_iv_i^2}{2}-\frac{m_jv_j^2}{2}
        =-\frac{\mu g^2}{2}=-\ltx{E}{internal},
\end{equation}
so, all the internal energy is lost during the aggregative collision.
The threshold energy value $\ltx{E}{agg}$ is in general a function of the sizes of 
particles.

Let us write the debris velocity distribution function for the aggregation process. 
Since the outcome is a single particle of mass $m_i+m_j$, with velocity $\bV$, we have
\begin{equation}
    \utx{P_k}{agg}\pqty{\bv}=\delta_{k,i+j}\delta\bqty{\bv-\mu_i\bv_i-\mu_j\bv_j}.
\end{equation}


\subsection{Fragmentation}
If the impact energy exceeds the certain threshold value 
$\ltx{E}{imp}\geqslant\ltx{E}{frag}$, the two impactors break into into smaller
particles in the collision.
We cannot obtain the velocities of the monomers from only conservation laws, hence 
we have to assume that certain constraints are valid. Namely, we assume two 
constraints:
\begin{enumerate}
    \item Both particles shatter into their constituent monomers;
    \item Complete isotropy of the momenta of the monomers in CoM frame;
\end{enumerate}
These two constraints allow us to write the outcome velocities of the fragmented pieces.
Let us write the energy needed to release a single monomer from a particle as $\gamma$.
Hence, the total energy needed for a complete decomposition of an aggregate of mass $m_k$
can be estimated as
\begin{equation}
    E_k = \gamma\cdot k.
\end{equation}
The fragmentation process of two particles of masses $m_i$ and $m_j$, with velocities
$\bv_i$ and $\bv_j$ can be then described as a decay of a single particle of mass 
$m_k=m_i+m_j$ with velocity $\bv_k=\bV=\mu_i\bv_i+\mu_j\bv_j$. The decay energy can 
be estimated as 
\begin{equation}
    \ltx{E}{decay}=\ltx{E}{imp}-\gamma\cdot k,
\end{equation}
which is the amount of energy which is equally distributed among all the shattered 
monomers. From this, we can see that the impact energy should be larger than 
$\gamma\cdot k$, which can be treated as the threshold energy. Since the impact energy 
is the normal part of the internal energy, we can write
\begin{equation}
    \ltx{E}{decay}=\frac{\mu g_n^2}{2}-\gamma\cdot k=\eps^2\frac{\mu g_n^2}{2},
\end{equation}
and the restitution coefficient for the fragmentation is
\begin{equation}
    \eps=\sqrt{1-\frac{2\gamma k}{\mu g_n^2}}.
\end{equation}
Since the decay energy has to be positive, we can write the threshold value for the 
normal relative velocity as 
\begin{equation}
    g_n\geqslant\sqrt{\frac{2\gamma k}{\mu}}=
    \sqrt{\frac{2\gamma}{m_1}}\cdot\frac{i+j}{\sqrt{ij}}.
\end{equation}
In the CoM frame, each released monomer has an energy 
\begin{equation}
    E'_{c}=\frac{m_1v'^2_{c}}{2}=\frac{\ltx{E}{decay}}{k}=\eps^2\frac{\mu g_n^2}{2k},
\end{equation}
where $v'_{c}$ is the speed of a monomer in CoM frame
\begin{equation}
    v'_{c}=\frac{\sqrt{ij}}{i+j}\cdot\eps g_n.
\end{equation}

Let us estimate the number of monomers $\dd{N}$ in a small solid angle $\dd{\Omega}$.
From the second constraint, we deduce that this number has to be proportional to the 
angle itself, hence 
\begin{equation}
    \dd{N}=\frac{k}{4\pi}\dd{\Omega},\qquad k=i+j.
\end{equation}

In the Lab frame, the speeds of monomers are not equal, but rather uniformly 
distribution in the range 
\begin{equation}
    \lmin{v'}=V-v'_c,\qquad\lmax{v'}=V+v'_c.
\end{equation}

Since the fragmented debris consist of only monomers, the distribution function 
$P_k\pqty{\bv}$ has to contain the term $\delta_{k,1}$. In the CoM frame, we can write 
\begin{equation}
    \utx{P_k}{frag, CoM}\pqty{\bv}=\delta_{k,1}\delta\pqty{v-v'_c}\frac{i+j}{4\pi}.
\end{equation}
In this case, the integral of any velocity function $\varphi\pqty{\bv}$ in the form of 
\begin{equation}
    \int\dd{\bv}\varphi\pqty{\bv}\utx{P_k}{frag,CoM}\pqty{\bv}=
    \delta_{k,1}\frac{i+j}{4\pi}\int\dd{\bv}\varphi\pqty{\bv}\delta\pqty{v-v'_c},
\end{equation}
can be written as 
\begin{equation}
    \delta_{k,1}\frac{i+j}{4\pi}\int\dd{\be}
    \int_{0}^{\infty}\dd{v}\varphi\pqty{v,\be}\delta\pqty{v-v'_c}=
    \delta_{k,1}\frac{i+j}{4\pi}\int\dd{\be}\varphi\pqty{v'_c,\be}.
\end{equation}
If $\varphi\pqty{\bv}\equiv\varphi\pqty{v,\be}=\be\varphi\pqty{v}$, such as $\bv=v\be$, 
then 
\begin{equation}
    \int\dd{\be}\be\varphi\pqty{v}=\bnull.
\end{equation}
If $\varphi\pqty{\bv}\equiv\varphi\pqty{v,\be}=\varphi\pqty{v}$, then 
\begin{equation}
    \int\dd{\be}\varphi\pqty{v}=4\pi\varphi\pqty{v}.
\end{equation}

To write the debris velocity distribution function in the Lab frame, we have to 
add the center of mass velocity to all the velocities of the monomers. This can be 
written as 
\begin{equation}
    \utx{P_k}{frag}\pqty{\bv}=
    \delta_{k,1}\frac{i+j}{4\pi}\int\dd{\be}\delta\pqty{\bv-\bV-v'_c\be}.
\end{equation}
Now, integrating over a function $\varphi\pqty{\bv}$ becomes
\begin{equation}
    \delta_{k,1}\frac{i+j}{4\pi}\int\dd{\be}\int\dd{\bv}\varphi\pqty{\bv}
    \delta\pqty{\bv-\bV-v'_c\be}=
    \delta_{k,1}\frac{i+j}{4\pi}\int\dd{\be}\varphi\pqty{\bV-v'_c\be}.
\end{equation}
If $\varphi\pqty{\bV-v'_c\be}=\varphi\pqty{\bV}-\be\varphi\pqty{v'_c}$, then 
we have
\begin{equation}
    \delta_{k,1}\frac{i+j}{4\pi}\int\dd{\be}\int\dd{\bv}\varphi\pqty{\bv}
    \delta\pqty{\bv-\bV-v'_c\be}=\delta_{k,1}\pqty{i+j}\varphi\pqty{\bV}.
\end{equation}


\section{Distribution function}
The statistical description of the system is fully described by a set of distribution
functions $f_k\pqty{\br,\bv,t}$. It is normalized, such that 
$f_k\pqty{\br,\bv,t}\dd{\br}\dd{\bv}$ gives the number of particles of size $k$
in the phase space volume $\dd{\Gamma}=\dd{\br}\dd{\bv}$, around the point 
$\pqty{\br,\bv}$. Hence, integrating over the whole phase space gives us the total
number of particles of size $k$
\begin{equation}
    N_k=\int\dd{\br}\dd{\bv}f_k\pqty{\br,\bv,t}.
\end{equation}
The spacial distribution of particles is not very important for us, hence in the following
we assume that the system is spatially homogeneous, and we use only the velocity
distribution function $f_k\pqty{\bv,t}$
\begin{equation}
    N_k=\int\dd{\br}\int\dd{\bv}f_k\pqty{\bv,t},
\end{equation}
hence 
\begin{equation}
    n_k\equiv\frac{N_k}{V}=\int\dd{\bv}f_k\pqty{\bv,t},
\end{equation}
is the number density of the subsystem of particles with size $k$. The other field 
functions, such as the mean flow velocity $\bu_k$ or granular temperature $T_k$ 
can be defined as velocity moments of the distribution function
\begin{equation}
    \begin{split}
        n_k\bu_k &= \int\dd{\bv}\bv f_k\pqty{\bv,t},\\
        \frac{3}{2}n_kT_k &=\int\dd{\bv}\frac{m_kc_k^2}{2}f_k\pqty{\bv,t},\\
        \bc_k &= \bv-\bu_k.
    \end{split}
\end{equation}

\section{Kinetic equations}
The time evolution of the distribution functions obey the Boltzmann equations
\begin{equation}
    \pqty{\pdv{t}+\bv\vdot\pdv{\br}-\frac{1}{m_k}\pdv{U(r)}{\br}\vdot\pdv{\bv}}
    f_k\pqty{\br,\bv,t}=\sum_{j}\mathcal{I}\pqty{f_k,f_j},
\end{equation}
where $U(r)$ is the potential of the external gravitational field. The LHS
of the Boltzmann equation describes the change over time in the function $f_k$ due to the
local flow of the particles, subject to external driving. The function 
$\mathcal{I}\pqty{f_k,f_j}$ on the RHS is the \emph{collision integral}, which 
describes the change over time in the function $f_k$ due to collisions of particles $k$
with particles of size $j$. Since we have three types of collisional outcomes, the 
collision integral $\mathcal{I}$ has to take into account all these types of outcomes.
Without the loss of generality, we can write the collision integral as a sum of three 
functions
\begin{equation}
    \mathcal{I}\pqty{f_k,f_j}=\utx{\mathcal{I}}{agg}\pqty{f_k,f_j}+
    \utx{\mathcal{I}}{res}\pqty{f_k,f_j}+\utx{\mathcal{I}}{frag}\pqty{f_k,f_j},
\end{equation}
each corresponding to the specific type of collision.

\subsection{General structure of collision integrals}
Let us consider a collision integral $\mathcal{J}\pqty{f_k,f_j}$ for a generalized 
collision.


    







\end{document}