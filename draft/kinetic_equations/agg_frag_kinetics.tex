\documentclass[aps,prl,preprint,groupedaddress,10pt]{revtex4-2}
\usepackage{notations}

\begin{document}
\section{The mechanics of binary interactions}
Our system comprises an infinite number of mechanically identical, spherical particles 
known as monomers. These particles have masses $m_1$ and radii $R_1$. Although their 
masses are quite small, they can interact with each other gravitationally. When two 
monomers collide, they lose a certain amount of impact energy and rebound with a 
coefficient of restitution $\eps$. If the impact energy is below a specific threshold 
value $\ltx{E}{imp}\leqslant\ltx{E}{agg}$, the monomers stick together due to surface 
forces like van der Waals forces, forming a larger aggregate particle with mass $m_2$ 
and radius $R_2$. This process is known as \emph{aggregation} and allows the creation 
of larger particles from individual monomers.

Conversely, there is another mechanism called \emph{fragmentation} that decreases the 
sizes of aggregates. If the impact energy is higher than a certain threshold value 
$\ltx{E}{imp}\geqslant\ltx{E}{frag}$, the colliding aggregates break into smaller pieces.

\subsection{Distribution function}
The statistical description of the system is given by 

\subsection{Collision mechanics}
We consider a collision of two particles of masses $m_i$, $m_j$, and velocities $\bv_i$, 
$\bv_j$. If the particles did not exert gravitational influence, the collision 
geometry would have been a linear problem. However, the gravitational interaction of 
the particles result in a deflection of the trajectories of motion, aka 
\emph{gravitational focusing}. To analyze the collision mechanics of two gravitationally 
interacting bodies, let us reformulate the problem into a mathematically equivalent one.
Consider a point size particle with mass $\mu=m_im_j/M$ and velocity $\bg_0=\bv_i-\bv_j$, 
that moves in a gravitational field of a stationary body of mass $M=m_i+m_j$ and radius 
$R=R_i+R_j$. The collision happens if the point mass falls onto the surface of the 
stationary body. Let us introduce a parameter $b$, which is a distance from the center of 
the stationary mass to the trajectory asymptote of the moving particle. The vector $\bb$
is then orthogonal to the velocity $\bg$. There is a certain value of this parameter 
$\lmax{b}$, which defines the collision threshold, e.g. when the impact parameter $b$
is smaller than this threshold value, the moving particle falls onto the surface of the 
stationary body. The value of this threshold value is
\begin{equation}
    \lmax{b}\pqty{g_0}=R\sqrt{1+\frac{2GM}{Rg_0^2}},
\end{equation}
where $G$ is the gravitational constant. The value of the relative velocity $g_0$ is 
considered at infinity, hence at the point of impact, the relative speed, or in other 
words, the impact speed $g$ is 
\begin{equation}
    g = g_0\sqrt{1+\frac{2GM}{Rg_0^2}},
\end{equation}
slightly larger than $g_0$. The point of impact on the surface of the stationary body 
can be represented by a unit vector $\be$, pointing from the center of the body. However,
the actual impact speed does not depend on $\be$, i.e. as long as the impact parameter 
of the point mass particle is less than the threshold value $\lmax{b}$, 
it falls onto the surface of the stationary body with the same speed $g$ 
for any value $b$ of the impact parameter, due to energy conservation and spherical shape
of the body.

\subsection{Aggregation mechanics}




    
\end{document}